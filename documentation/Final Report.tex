\documentclass[conference]{IEEEtran}
\IEEEoverridecommandlockouts
% The preceding line is only needed to identify funding in the first footnote. If that is unneeded, please comment it out.
\usepackage{cite}
\usepackage{amsmath,amssymb,amsfonts}
\usepackage{algorithmic}
\usepackage{graphicx}
\usepackage[utf8x]{inputenc} 
\usepackage[OT1]{fontenc} 
\usepackage{textcomp}
\usepackage{xcolor}
\usepackage{mathtools}
\usepackage{subfig}
\usepackage{hyperref}

\def\BibTeX{{\rm B\kern-.05em{\sc i\kern-.025em b}\kern-.08em
    T\kern-.1667em\lower.7ex\hbox{E}\kern-.125emX}}
\begin{document}

\title{Understanding Image Advertisements and Predicting Sentiment\\
{\footnotesize \textbf{Github Link:} \url{https://github.com/rohanchopra/processing-ads}}
}

\author{\IEEEauthorblockN{1\textsuperscript{st} Rohan Chopra}
\IEEEauthorblockA{\textit{40233019} \\
\textit{Concordia University}\\
Montreal, Canada \\
rohanchopra09@gmail.com}
\and
\IEEEauthorblockN{2\textsuperscript{nd} Harman Singh Jolly}
\IEEEauthorblockA{\textit{40204947} \\
\textit{Concordia University}\\
Montreal, Canada \\
jollyharmansingh@gmail.com}
}


\maketitle

\begin{abstract}
Image based advertisements are still one of the best ways to promote products 
but it is painstakingly difficult to personalize the content for the target audience 
and covey the sentiments. This study tries to compare three backbone deep learning 
architectures namely, ResNet 50, MobileNetv3 Large and EfficientNet B3 
on an image advertisement dataset to classify the underlying sentiments 
being perceived by the consumers. Transfer learning is used to mitigate the small 
dataset problem.

EfficientNet performed the best overall but the performance was still very poor. Grad-CAM visualizations
confirmed our understanding of this model and helped us gain more confidence on the performance of the model.
\end{abstract}

\begin{IEEEkeywords}
CNN, advertisement, image, Grad-CAM, deep learning, transfer learning, Efficientnet, Mobilenet, Resnet, multilabel, image classification
\end{IEEEkeywords}

\section{Introduction}
In today's data centric world, companies are looking to improve their processes 
 with the ever increasing amount of data we generate. With this data, comes 
 along user profiling and understanding consumer emotions. The 
 expansion of online visual content and social media has led to a surge of 
 interest in the large-scale social multimedia analysis. For a long time image advertisements 
have been a prevalent part of our daily lives. From billboards to social 
media ads, companies have been constantly vying for our attention through visually 
appealing images. Traditionally, only a few subject matter experts understood how 
an advertisement would be perceived by consumers. However, with the rise of 
convolutional neural networks, it is now possible to predict the emotions that 
these consumers face when they look at these images. So, 
pondering over the platitude, `A picture says a thousand words' becomes necessary.

Tech giants like Facebook and Google have produced enormous wealth by 
advertisements. 97.5\% of the \$116.6 billion revenue generated by 
Facebook\cite{b1} and 70.9\% of Google's revenue\cite{b2} came in from 
broadcasting advertisements. Attracting customers to a product and producing a 
certain emotion is a vital aspect of using advertisements on television, social media
 and billboards. Selection of ads by hand is a physically tedious and onerous job 
 where provision of image based advertisements or videos becomes highly difficult 
 and subjective via handpicking. 
So, automatic advertising techniques such as the contextual 
advertising method was developed which aims to find the most relevant advertisements
 for a content that would annoy customers the least.

The components that contribute to the effectiveness of these advertisements are 
intricate and are being analyzed in marketing science and consumer psychology. 
A particular focus of this research has been on comprehending the emotions and 
sentiments conveyed by visual media content, which has become increasingly 
popular for both academic and industry purposes. Sentiment analysis is primarily the 
process of identifying and extracting opinions, emotions, and attitudes expressed 
in text. However, with the increasing use of visual media in advertising, a need 
for sentiment analysis on images has come up. Sentiment labels are used to 
annotate images with positive, negative or neutral sentiments. These annotations 
are used to train machine learning models to predict the emotions that an image is 
likely to evoke in viewers.
Powerful emotions conveyed by images and videos can amplify the message conveyed 
in the content, making it more impactful and capable of influencing the audience 
more effectively. But sometimes, advertisements aren't able to hold viewers 
interest as the product shown doesn't necessarily incline to the viewers choices 
or viewers actually skip watching advertisements that aren't able to garner 
their attention, which is where a positive or a negative connotation comes 
into picture pertaining to sentiment of an advertisement.

In this study, we try to identify emotions that people felt when they looked 
at image advertisements using convolutional neural networks. Due to the small 
size of data and similarity to ImageNet, transfer learning was used to train the model.
This report is organized as follows: Section 2 discusses similar works in the literature, 
section 3 describes the dataset and the methodology used in this study, section 4 
presents the results with the experimental setup and section 5 concludes the report with 
some suggestions for future work.

\section{Related Works}

In order to create an effective advertisement, researchers take into account a 
ton of ideas to reach a balance of emotion and the type of message to convey. Common 
sense in humans dictates that the “red color” symbolizes “stop” when used in 
traffic related series and symbolizes “blood” when used in medical related settings 
but, computers have no way of knowing this information. Advertisements use different 
types of visual rhetorics like the color red to convey their message and invoke 
emotions in the consumer. To fathom these rhetorics it is required to discern
the objects in the advertisement and then decode this rhetorics from all these 
objects \cite{b4}~\cite{b5}. Caption generation is one of the use cases where 
 a summarization about the image is required. Different studies try to understand 
 \cite{b6}~\cite{b7} which objects are being portrayed in an image, how they 
are portrayed and answering the main question: why are they being portrayed. 
For instance, if an advertisement is meant to target a young audience, it can be 
designed with bright colors and positive sentiments to appeal to that demographic. 
On the other hand, an ad meant for an older audience may be designed with muted 
colors and neutral sentiments. By predicting the sentiments that an advertisement 
is likely to evoke, companies can tailor their ads to specific audiences, increasing 
their chances of success.

Exploring objects or generic nouns such as “cat” or “trees” have been a marvel of 
machine vision but the association of sentiments correlating to the visuals remains 
a challenging and seemingly insurmountable task. The difficulty of this endeavor 
stems from the significant emotional distance between the low visual characteristics 
and the high-level sentiment that adjectives convey. To fill this gap between visual 
characteristics and sentiments, Borth, Damian et al.\cite{b8} proposed an alternative 
method that represents sentiment-related visual concepts for an intermediary 
representation. They have coined the term Adjective Noun Pairs (ANP) like "sprawling trees" 
or "sleepy cat" which are used to corroborate the adjective of nouns, further used to 
identify nouns. ANPs, despite not conveying emotions or sentiments explicitly, can 
still be effective indicators for identifying emotions portrayed in images, as they 
were identified based on a high correlation with emotion tags present in web photos. 
In their study Borth, Damian et al.\cite{b8} trained binary SVM classifiers on these 
ANPs for whole images. To further enhance the classifiers, the authors Chen, Felix 
et al.\cite{b9} included the localization of object-based concepts and the semantic 
similarity between these concepts in their work.

DeepSentiBank\cite{b10} shows significant efforts to interpret emotion from images 
using a model trained on web images that are tagged with multiple sentiments. Hussain, 
Zhang et al.\cite{b11} went on to apply DeepSentiBank on advertisements and reached a conclusion 
that to a detector for natural images (DeepSentiBank) couldn't be applicable for 
advertisement images. Besides DeepsentiBank, Vedula \& Sun et al.\cite{b12} developed 
an advertisement recommendation system using sentiments in multimedia content. Using 
deep learning techniques, image and video analysis together, has garnered a great 
amount of attention at Youtube for contextual understanding in video advertising\cite{b13}. 
But, compared to other forms of media such as publishing 
or billboards, which are less personalized, a great deal of work lies in interpreting 
advertisements media due to their extensive use of visual rhetoric as coined in 
Hussain, Zhang et al.\cite{b11}. 

An advertisement demands much in depth analysis due to the fact that they intend to 
persuade people to either buy or use a perticular product or, to make people aware 
of a social cause. The motive to persuade people is deeply imbibed 
inside an advertisement, where sometimes it is portrayed via simple tone or sometimes 
via sarcasm. In order to understand this underlying sentiment, only appearance isn't 
enough. Therefore, to personalize the experience with ads, it is imperative to not 
only understand its topic, but also the emotion it conveys. Zhang, Luo et al.\cite{b14} 
introduced a framework that amalgamates multiple modalities together for envisioning 
topic and sentiment related predictions to further conceptualize advertisements.

\section{Dataset} \label{sec:dataset}

We have used a publicly available dataset developed by the combined efforts of 
Hussain et al. at the University of Pittsburgh with over 64,000 advertisement images 
and over 3,000 video advertisements. The authors used Amazon Mechanical Turk workers
 to tag each advertisement to its respective topic (eg.\ category of the product  
 the advertisement targets) and what sentiment it conveys to the viewer (eg.\ how plants/trees 
play a vital role in sustenance) followed by what method it uses to imbibe that 
message (eg.\ the presence of trees or plants might be depicting life). The approach 
used to gather and annotate this data was influenced by the research in Media Studies, 
an academic field that examines the content of mass media messages, with input from 
one of the research paper authors\cite{b11} who had formal education in the 
field. The data is accessible at http://www.cs.pitt.edu/~kovashka/ads/.


\begin{figure}[htbp]
\includegraphics[width=8cm, height=10cm, keepaspectratio]{figures/dataset.eps}
\caption{Sentiment Label Distribution}
\label{fig:dataset}
\end{figure}

\section{Methodology} \label{sec:prop_method}

Before training, the images were analysed to come up with a pre-processing pipeline 
to denoise the images and improve their quality as most of the images were highly 
compressed.

\subsection{Bilateral Filter}
For improving the quality of the images, a smoothing filter for images had to be 
employed. So, a bilateral filter was used to reduce noise while preserving edges 
in a non-linear manner. This filter works by calculating a weighted average of 
intensity values from surrounding pixels to replace the intensity of each pixel. 
The weight of each pixel is determined using a Gaussian distribution and it preserves 
sharp edges. The weights used in the filter are not solely based on the distance 
between pixels, but also take into consideration differences in radiometric properties 
such as color intensity and depth distance\cite{b15}.

\begin{figure}[htp]
\includegraphics[width=9cm, height=11cm, keepaspectratio]{figures/ads_bilateral.eps}
\caption{Bilateral Filter applied on Ads }
\label{fig:ads_bilateral}
\end{figure}

A bilateral filter is controlled by three parameters sigma(s), responsible for spatial 
region which smooths larger surfaces as we increase the spatial parameter. Sigma(r) 
which tends to act like a Gaussian filter when it increases as the Gaussian range 
widens and d, which is the diameter of each pixel neighbourhood. It is quintessential 
to know that all other filters smudge the edges, while Bilateral Filtering retains 
them. When weights are multiplied together and if one of the weights is close to
 zero, no smoothing occurs. This can be demonstrated by using a large spatial Gaussian 
 with a narrow range Gaussian, which results in limited smoothing despite the filter 
 having a large spatial extent. The range weight ensures that the edges are preserved.

As shown in Fig.~\ref{fig:ads_bilateral}, after applying different values of d, sigma(s) 
and sigma(r) on different images, it was determined that the pixel diameter of 9 
along with sigma(s) and sigma(r) being 9 gave the best results. Following that, 
the bilateral filter has been applied twice merely because of the fact that applying 
bilateral filters in iterations enhanced picture quality even more. 
Fig.~\ref{fig:ads_bilateral} depicts a significant change in the filtered image 
when compared with the original image. The original image had color banding or posterization,
 an ugly artifact that can be seen in digital images around objects. This has been notably 
 reduced\cite{b16} and the final image is better than previous one. Another remarkable 
 change that was witnessed was the compression artifacts, a distortion of media in images 
 which is caused by lossy compression of media was also removed in the 
 final image corroborating the efforts. 

\subsection{Pre-processing Techniques}
Pytorch provides various functional transformations that can be applied using the 
torchvision.transform module. They accept both PIL images and tensor images, 
although some transformations are PIL-only and some are tensor-only\cite{b17}. 
As these transformations require a parameter such as a factor by which an image 
can be transformed, therefore they cannot be applied to all images owing to the 
fact that all images are different.

\begin{figure}[htbp]
    \includegraphics[width=9cm, height=9cm, keepaspectratio]{figures/preprocessing_plots.eps}
    \caption{Pytorch Preprocessing Techniques}
    \label{fig:preprocessing_plots}
    \end{figure}

For example, a Hue transform accepts an image along with a parameter, hue\_factor 
that ranges from [-0.5 to 0.5]. 0.5 and -0.5 give complete reversal of the hue channel 
in HSV space in positive and negative direction respectively whereas 0 means no shift. 
The same level of factor cannot be expected from other techniques such as sharpness 
or contrast etc. Hence, this parameter cannot be kept constant for all images as it 
will have a variable effect or appeal on different images. In the Fig.~\ref{fig:preprocessing_plots}, 
five random images have been selected and functional image processing techniques like 
hue transforms, gamma transforms, solarize transformations, sharpness, etc 
have been applied to reach a conclusion that all images bearing uniqueness in their 
characteristics respond differently to functional transformations applied.

Even though brightness and contrast change show a potent outcome, a specific parameter 
cannot be kept for all images So, this processing approach has been handled in data 
augmentation section where functions like AutoContrast and AutoBrightness have been applied. 
Histogram equalization, a popular technique for improving the contrast of images visibly 
failed to work for colored images. This technique was employed on a colored image which 
distorted colors and features such that it could not be considered as a viable pre-processing technique.

\begin{figure}[htbp] 
    \includegraphics[width=9cm, height=9cm, keepaspectratio]{figures/histogram_equalize.eps} 
    \caption{Histogram Equalization on Colored Image} 
    \label{fig:histogram_equalize}
    \end{figure}

Fig.~\ref{fig:histogram_equalize} shows the aftermath of histogram equalization on a 
colored image in detail where the image has been completely distorted in it's entirety 
and is difficult for any feature to be mapped.

\subsection{Data Augmentations Techniques}
Data augmentation plays a crucial role in the training of deep 
learning models as they aren't able to converge the network to an optimal 
solution if the size of training data is small because of the huge number of parameters 
needed to be tuned by the learning algorithm. 
It requires enormous amounts of data merely because of the fact that the deep learning 
algorithms start off with a poor initial state where weights are completely random 
and then optimization occurs using some gradient based optimization algorithm. 
There are various ways to augment data using the PyTorch library such as 
RandomHorizontalFlip, RandomAdjustSharpness, etc that produce images with any 
random factor while training the network. Data augmentation also helps in creating images
 that may be a possible in the real world but are not depicted in the dataset properly.
 For example, a coffee mug might be perfectly straight in one image but it could be 
 a tilted by 15 degrees in another image. RandomRotation helps generate such images.
 In Fig.~\ref{fig:RandomHorizontalFlip},~\ref{fig:RandomRotation30},~\ref{fig:ColorJitter},
 ~\ref{fig:RandomAutocontrast} and~\ref{fig:RandomAdjustSharpness} 
 data augmentation being used in this study has been visualized.
 RandomHorizontalFlip, RandomRotation to a maximum of 30 degrees, RandomColorJitter, 
 RandomAutocontrast with factor of 5 and lastly RandomAdjustSharpness with sharpness factor as 2 
 has been used.

\begin{figure}[htbp] 
    \includegraphics[width=8cm, height=9cm, keepaspectratio]{figures/RandomHorizontalFlip.eps} 
    \caption{Data Augmentation using RandomHorizontalFlip} 
    \label{fig:RandomHorizontalFlip} 
    \end{figure}

\begin{figure}[htbp] 
    \includegraphics[width=8cm, height=9cm, keepaspectratio]{figures/RandomRotation30.eps} 
    \caption{Data Augmentation using RandomRotation} 
    \label{fig:RandomRotation30}
    \end{figure}

\begin{figure}[htbp] 
    \includegraphics[width=8cm, height=9cm, keepaspectratio]{figures/ColorJitter.eps} 
    \caption{Data Augmentation using ColorJitter} 
    \label{fig:ColorJitter} 
    \end{figure}

\begin{figure}[htbp] 
    \includegraphics[width=8cm, height=9cm, keepaspectratio]{figures/RandomAutocontrast.eps} 
    \caption{Data Augmentation using RandomAutocontrast} 
    \label{fig:RandomAutocontrast}
    \end{figure}

\begin{figure}[htbp] 
    \includegraphics[width=8cm, height=9cm, keepaspectratio]{figures/RandomAdjustSharpness.eps} 
    \caption{Data Augmentation using RandomAdjustSharpness} 
    \label{fig:RandomAdjustSharpness}
    \end{figure}
    

\subsection{Model Architectures} 
(\ref{tab:selArch}) Different backbone architectures were chosen to ensure that different 
types of Convolution blocks were tested for the advertisement data. Other selection criteria 
included the \textit{number of parameters} and \textit{GFLOPS}, important to keep track of 
the total training and evaluation time, and the \textit{top 5 classification accuracy} on the 
ImageNet 1K benchmark dataset. Finally, the following three
backbone architectures were chosen:

\begin{table}[htbp]
    \caption{Shortlisted Backbone Architectures.}
    \centering
    \begin{tabular}{p{1.5cm}|p{1cm}|p{1cm}|p{1.2cm}|p{1cm}}
    %\toprule
    \hline
    Arch. & Params (Mil.) & Layers & GFLOPS & Imagenet Acc.\\
    %\midrule
    \hline
    MobileNet V3 Large & 5.5 & 18 & 0.22 & 92.57\\
    %\midrule
    \hline
    EfficientNet B3 & 12.2 & 29 & 1.83 & 96.05\\
    %\midrule
    \hline
    Resnet 50 & 25.6 & 50 & 4.09 & 95.43\\
    %\bottomrule
    \hline
    \end{tabular}
    %\vspace{-1.5em}
    \label{tab:selArch}
  \end{table}

\textbf{ResNet 50}: A residual learning CNN with 50 layers that  are made possible by skip connections. Without these skip  connections, training such a deep network is not possible due to the vanishing gradient problem. The 50 layer variant was chosen to decrease training time while not compromising on the accuracy. This architecture had the highest trainable parameters, FLOPS and number of layers\cite{b18}.

\textbf{MobileNet V3 Large}: This model uses depthwise separable convolution  from MobileNet V2 along with squeezeexcitation blocks in residual layers  from MnasNet. This makes it really quick to train while still performing at par with other architectures. This architecture had the lowest trainable parameters and FLOPS among the three selected. Howard et al. Reference\cite{b19} also used network architecture search to find the most effective model. The large configuration was chosen to not compromise on the prediction accuracy.

\textbf{EfficientNet B3}: This model uses compound scaling to scale the model by 
depth, width and resolution. The B3 version was chosen to have faster training without compromising on the accuracy. Reference\cite{b20} This architecture performs the 
best among the selected on the Imagenet benchmark dataset while having half  the trainable parameters of Resnet50.


\subsection{Optimization Algorithm}
The Adam optimizer \cite{b21} is an adaptive learning rate optimization algorithm which was chosen as the optimizer for this study as it converges faster by integrating benefits of the RMSProp algorithm and momentum technique. It is also robust to hyperparameters but, requires tweaking of the learning rate depending on the task at hand. For this study, a learning rate of 0.01 and the original author recommend settings for $\beta_{1} = 0.9$, $\beta_{2} = 0.999$ and $\epsilon = 10^{-8}$ were used for the first and second order moment estimate as defined in \eqref{eq:adam1} and \eqref{eq:adam2} where $\beta_{1}$ and 
$\beta_{2}$ control the decay rates.

 \begin{equation}
 m_{t} = \beta_{1} \cdot m_{t-1} + (1 - \beta_{1}) \cdot g_{t}
 \label{eq:adam1}
 \end{equation}

 \begin{equation}
   v_{t} = \beta_{2} \cdot v_{t-1} + (1 - \beta_{2}) \cdot g_{t}^{2}
   \label{eq:adam2}
   \end{equation}

Further, the Cosine annealing \cite{b22} learning rate scheduler was used 
to reduce the learning rate as the training progressed down to an end point of 0.001. This 
would help us reduce the learning rate as training progressed, preventing us from overshooting the minima.

\section{Results} \label{sec:result}
\subsection{Experiment Setup}

First, training data with labels that were low in count were removed. Out of the 30 available
labels, only the top 20 were chosen. Next, images where the label was categorised by only one 
person was discarded as it was observed that many of these images were incorrectly labeled.

Then, the advertisement images were pre-processed using the bilateral filter described 
in \ref{sec:prop_method} before resizing them to the 
size - 384 * 384. The data was split into train, test and validation set in the 0.6:0.2:0.2 
ratio and was stored in separate directories according to the defined PyTorch dataloader. 
Then, the mean and standard deviation of the dataset was calculated using the training dataset. 
All images were normalized before training using the dataloader with this calculated mean and 
standard deviation. 

The dataset used in this study presented the the multiclass, 
multilabel classification problem.  Thus, to make the model predict multiple labels, a 
sigmoid layer had to be added before the loss function to get 0 or 1 prediction for 
all the classes of the data. To achieve this, the BCEWITHLOGITSLOSS function of PyTorch 
was used as it combines the Sigmoid layer and the  binary cross entropy loss function 
in one single class. This makes theses operations more numerically stable than their 
separate counterparts \cite{b23}.

The backbone architectures and their pre-trained weights were obtained directly from the 
torchvision library and the final classification layer was modified for our dataset of 
20 classes. The pre-trained weights were chosen to be the IMAGENET1K\_V2 weights and only 
the last classification layer was fine-tuned. The rationale behind performing this type of 
shallow-tuning was that the Imagenet data is very similar to the advertisement images in 
our dataset. Additionally, the size of the selected dataset is small so deep-tuning might not work well. 

The batch size was fixed to 32 for all the models. While training, the best model by 
validation loss was saved to prevent the usage of overfit models for the test set analysis. 
The actual and predicted results from each epoch was also stored to calculate the F1 scores at 
each step of training. While calculating the F1 score, macro averaging 
was used to get an average score across classes. 

Initial training runs of the multilabel data produced a zero F1 score due to its highly imbalanced
 nature. To mitigate this, class wise weights were calculated and used with the loss function. 
 This improved the F1 score quite considerably.

\begin{figure}[htbp]
    \centering
    \includegraphics[width=1\linewidth]{figures/metrics.eps}  
    \caption{Training \& Validation, F1 \& Loss plots for the three models.}
    %\vspace{-1em}
    \label{fig:acc_loss}
  \end{figure}


Finally, the best models from each of the three training runs validation loss were used 
to get the test set metrics that are displayed in \ref{tab:test_metrics}. 
Per epoch training and validation F1 score and loss are provided in 
Fig.~\ref{fig:acc_loss}. Next, the best model which in this case was the EfficientNet B3 model 
was used to do further analysis like visualizing the trained filters and using Grad-CAM to 
understand which areas of the image the model focused on to generate the predictions. 

\subsection{Training Results}
From Fig.~\ref{fig:acc_loss} it is clear that going from a smaller 
architecture to a bigger architecture, makes the model start to overfit earlier. 
The MobileNet model took the most number of epochs to reach the minima. The EfficientNet model 
performs the best for our dataset. However, all three models performed poorly. This shows that the 
compound scaling of EfficientNet gives good results for the advertisement dataset. The reason for 
poor performance overall could be due to a number of reasons. 
The dataset had a high number of classes but the number of examples per class was very low.
Moreover, there was a high class imbalance problem. After looking at the images and 
the labels more closely, it was noticed that many images were poorly labeled and the 
labels contained quite a few synonyms. 

\begin{table}[htbp]
    \caption{F1 (higher is better), time per epoch in seconds (lower is better), and number of epochs to reach the best validation loss and F1 (lower is better) for the 3 models that were trained.}
    \centering
    \boldmath
    \begin{tabular}{p{2cm}|p{1cm}|p{1cm}|p{1.2cm}|p{1cm}}
    %\toprule
    \hline
    Model & \emph{F1} & \emph{Time} & \emph{F1 Epochs} & \emph{Loss Epochs}\\
    %\midrule
    \hline
    MobileNet V3 Large & 0.168 & \textbf{80s} & 50 & 98\\
    %\midrule
    \hline
    EfficientNet B3 & \textbf{0.189} & 153s & \textbf{5} & 90\\
    %\midrule
    \hline
    Resnet 50 & 0.179 & 89s & 10 & \textbf{0}\\
    %\bottomrule
    \hline
    \end{tabular}
    %\vspace{-1.5em}
    \label{tab:test_metrics}
    \end{table}

    \begin{figure}[htbp]
      \centering
      \includegraphics[width=1\linewidth]{figures/efficientnet_cm.eps}  
      \caption{Confusion Matrix for the EfficientNet model.}
      %\vspace{-1em}
      \label{fig:conf}
    \end{figure}
  

Looking at \ref{tab:test_metrics} it can be seen that the MobileNet architecture was 
the fastest to train per epoch. It took less time per epoch but, if number of epochs 
required to converge is considered, it does not train the fastest. The lowest validation
 set loss for ResNet was at epoch 0. This means that the model started overfitting right 
 after the first epoch in terms of the loss. However, it took 10 epochs to converge on the F1 score. 
 EfficientNet model performed the best in terms of the overall F1 score on the test set. Another 
 surprising observation is that the EfficientNet model takes the longest to train per epoch even 
 though the number of trainable parameters is nowhere close to ResNet. 

 \begin{figure}[htbp]
  \centering
  \includegraphics[width=1\linewidth]{figures/first_layer_filters.eps}  
  \caption{Filters of the first layer.}
  %\vspace{-1em}
  \label{fig:filters}
\end{figure}

 Surprisingly, MobileNet isn't as fast to train as expected when compared to ResNet even though 
it has five times the learnable paramenters. This could be due to two reasons, 1. depthwise 
convolutions are not optimized in the version of PyTorch and CUDA used and 2. training is 
getting CPU bound due to the data augmentation before each training run which would take the 
same amount of time for all the models. 


In Fig.~\ref{fig:conf} it can be seen that the model classifies the labels Fashionable, Feminine, and Eager
 the best which are the classes that have the most number of training examples. This shows that if we 
 increase the training dataset size, the models could improve a lot.

\begin{figure}
    
    \subfloat[First Layer Activations]{\includegraphics[width = 1\linewidth]{figures/1_Layer_first.eps}}\\
    \subfloat[Middle Layer Activations]{\includegraphics[width = 1\linewidth]{figures/1_Layer_middle.eps}}\\
    \subfloat[Last layer Activations]{\includegraphics[width = 1\linewidth]{figures/1_Layer_last.eps}}
    \caption{GradCAM visualization for an Advertisement using the Efficientnet Model. Actual label - Eager}
    \label{fig:gradcam_1}
  \end{figure}



As the EfficientNet B3 model produced the best F1 score on the test set, it was used 
to generate the GradCAM visualizations to understand the model output. 
In Fig.~\ref{fig:gradcam_1} we can see that in the first layer of the network the model 
identifies prominent edges of the image. We can confirm this by looking at the filters 
of the first layer in Fig.~\ref{fig:filters}. Most of the filters look like they identify 
edges and corners. In the middle layer of the network, the model is looking at many 
different features but isn't looking at the most relavant features for that label. On 
the other hand, in the final layer of the network, the model looks only at the relavant 
features of the image depending on the current label. For example, here it is focusing 
on the player playing football for the 'Active' label. 

\begin{figure}
    
    \subfloat[First Layer Activations]{\includegraphics[width = 1\linewidth]{figures/2_Layer_first.eps}}\\
    \subfloat[Middle Layer Activations]{\includegraphics[width = 1\linewidth]{figures/2_Layer_middle.eps}}\\
    \subfloat[Last layer Activations]{\includegraphics[width = 1\linewidth]{figures/2_Layer_last.eps}}
    \caption{GradCAM visualization for an Advertisement using the Efficientnet Model. Actual label - Fashionable, Feminine, Persuaded}
    \label{fig:gradcam_2}
  \end{figure}

In Fig.~\ref{fig:gradcam_2}, the model focuses on the picture of the woman for the 'Feminine', 'Fashionable' and 'Cheerful' labels.

\begin{figure}
    
    \subfloat[First Layer Activations]{\includegraphics[width = 1\linewidth]{figures/9_Layer_first.eps}}\\
    \subfloat[Middle Layer Activations]{\includegraphics[width = 1\linewidth]{figures/9_Layer_middle.eps}}\\
    \subfloat[Last layer Activations]{\includegraphics[width = 1\linewidth]{figures/9_Layer_last.eps}}
    \caption{GradCAM visualization for an Advertisement using the Efficientnet Model. Actual label - Persuaded}
    \label{fig:gradcam_9}
  \end{figure}

In Fig.~\ref{fig:gradcam_9} We can see that the model correctly looked at the text 'Free' and classified the image as Thrifty however the actual label of persuaded was also in the top 4 predictions. This shows us that the model performs much better if we look at the images and generated outputs. The labels of the selected dataset are not completely correct making it difficult to evaluate the actual performance of the models. Visually looking at the gradcam visualizations and the predictions it is clear that the model is performing much better than what the F1 scores show.




\section{Conclusion}

Understanding how humans perceive image advertisements could help improve the quality of 
these advertisements. Training a neural network model to predict the emaotions 
felt by humans towards different images is easy buy collecting the data is much 
more difficult. The low performance of the models in this study can be attributed 
to the low quality of the labels along with a lack of available training data. 
Even though the convolutional layers of the EfficientNet model were not fine-tuned, 
it was observed that the model could find relavant features in the image depending on 
the label. This shows that transfer learning is a powerful tool to train models and reduce 
turn-around times. Transfer learning enables the use of deep learning models even when the 
amount of available data is very less. 

\begin{figure}
    
  \subfloat[First Layer Activations]{\includegraphics[width = 1\linewidth]{figures/7_Layer_first.eps}}\\
  \subfloat[Middle Layer Activations]{\includegraphics[width = 1\linewidth]{figures/7_Layer_middle.eps}}\\
  \subfloat[Last layer Activations]{\includegraphics[width = 1\linewidth]{figures/7_Layer_last.eps}}
  \caption{GradCAM visualization for an Advertisement using the Efficientnet Model. Actual label - Creative}
  \label{fig:gradcam_7}
\end{figure}

To improve the performance of the models, a number of things can be tried. 
Improving the labels of the available dataset can improve the models quite a bit. 
Additionally, getting more data would help the training effort by allowing the models 
to find patterns better. A long text description of how people feel when looking at 
different images could be an interesting take on this problem. Later, this text could 
be pre-processed using NLP models to create labels. 

\begin{thebibliography}{00}
\bibitem{b1} https://www.oberlo.com/statistics/facebook-ad-revenue

\bibitem{b2} https://www.statista.com/statistics/266249/advertising-revenue-of-google/

\bibitem{b3} Mei, Tao, Xian-Sheng Hua, Linjun Yang, and Shipeng Li. "VideoSense: towards effective online video advertising." In Proceedings of the 15th ACM international conference on Multimedia, pp. 1075-1084. 2007.

\bibitem{b4} Girshick, Ross. "Fast r-cnn." In Proceedings of the IEEE international conference on computer vision, pp. 1440-1448. 2015.

\bibitem{b5} Szegedy, Christian, Wei Liu, Yangqing Jia, Pierre Sermanet, Scott Reed, Dragomir Anguelov, Dumitru Erhan, Vincent Vanhoucke, and Andrew Rabinovich. "Going deeper with convolutions." In Proceedings of the IEEE conference on computer vision and pattern recognition, pp. 1-9. 2015.

\bibitem{b6} Y. Donahue, Jeffrey, Lisa Anne Hendricks, Sergio Guadarrama, Marcus Rohrbach, Subhashini Venugopalan, Kate Saenko, and Trevor Darrell. "Long-term recurrent convolutional networks for visual recognition and description." In Proceedings of the IEEE conference on computer vision and pattern recognition, pp. 2625-2634. 2015.

\bibitem{b7} Vinyals, Oriol, Alexander Toshev, Samy Bengio, and Dumitru Erhan. "Show and tell: A neural image caption generator." In Proceedings of the IEEE conference on computer vision and pattern recognition, pp. 3156-3164. 2015.

\bibitem{b8} Borth, Damian, Rongrong Ji, Tao Chen, Thomas Breuel, and Shih-Fu Chang. "Large-scale visual sentiment ontology and detectors using adjective noun pairs." In Proceedings of the 21st ACM international conference on Multimedia, pp. 223-232. 2013.

\bibitem{b9} Chen, Tao, Felix X. Yu, Jiawei Chen, Yin Cui, Yan-Ying Chen, and Shih-Fu Chang. "Object-based visual sentiment concept analysis and application." In Proceedings of the 22nd ACM international conference on Multimedia, pp. 367-376. 2014.

\bibitem{b10} Chen, Tao, Damian Borth, Trevor Darrell, and Shih-Fu Chang. "Deepsentibank: Visual sentiment concept classification with deep convolutional neural networks." arXiv preprint arXiv:1410.8586 (2014).

\bibitem{b11} Hussain, Zaeem, Mingda Zhang, Xiaozhong Zhang, Keren Ye, Christopher Thomas, Zuha Agha, Nathan Ong, and Adriana Kovashka. "Automatic understanding of image and video advertisements." In Proceedings of the IEEE conference on computer vision and pattern recognition, pp. 1705-1715. 2017.

\bibitem{b12} Vedula, Nikhita, Wei Sun, Hyunhwan Lee, Harsh Gupta, Mitsunori Ogihara, Joseph Johnson, Gang Ren, and Srinivasan Parthasarathy. "Multimodal content analysis for effective advertisements on youtube." In 2017 IEEE international conference on data mining (ICDM), pp. 1123-1128. IEEE, 2017.

\bibitem{b13} Madhok, Rishi, Shashank Mujumdar, Nitin Gupta, and Sameep Mehta. "Semantic understanding for contextual in-video advertising." In Proceedings of the AAAI Conference on Artificial Intelligence, vol. 32, no. 1. 2018.

\bibitem{b14} Zhang, Huaizheng, Yong Luo, Qiming Ai, Yonggang Wen, and Han Hu. "Look, read and feel: Benchmarking ads understanding with multimodal multitask learning." In Proceedings of the 28th ACM International Conference on Multimedia, pp. 430-438. 2020.

\bibitem{b15} https://en.wikipedia.org/wiki/Bilateral\_filter

\bibitem{b16} https://www.willgibbons.com/color-banding/

\bibitem{b17} https://pytorch.org/vision/stable/transforms.html

\bibitem{b18} He, Kaiming, Xiangyu Zhang, Shaoqing Ren, and Jian Sun. "Deep residual learning for image recognition." In Proceedings of the IEEE conference on computer vision and pattern recognition, pp. 770-778. 2016.

\bibitem{b19} Howard, Andrew, Mark Sandler, Grace Chu, Liang-Chieh Chen, Bo Chen, Mingxing Tan, Weijun Wang et al. "Searching for mobilenetv3." In Proceedings of the IEEE/CVF international conference on computer vision, pp. 1314-1324. 2019.

\bibitem{b20} Tan, Mingxing, and Quoc Le. "Efficientnet: Rethinking model scaling for convolutional neural networks." In International conference on machine learning, pp. 6105-6114. PMLR, 2019.

\bibitem{b21} Kingma, Diederik P., and Jimmy Ba. "Adam: A method for stochastic optimization." arXiv preprint arXiv:1412.6980 (2014).

\bibitem{b22} Loshchilov, Ilya, and Frank Hutter. "Sgdr: Stochastic gradient descent with warm restarts." arXiv preprint arXiv:1608.03983 (2016).

\bibitem{b23} https://pytorch.org/docs/stable/generated/torch.nn.BCEWithLogitsLoss.html

\end{thebibliography}

\end{document}
